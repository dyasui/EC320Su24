% Options for packages loaded elsewhere
\PassOptionsToPackage{unicode}{hyperref}
\PassOptionsToPackage{hyphens}{url}
\PassOptionsToPackage{dvipsnames,svgnames,x11names}{xcolor}
%
\documentclass[
  letterpaper,
  DIV=11,
  numbers=noendperiod]{scrartcl}

\usepackage{amsmath,amssymb}
\usepackage{iftex}
\ifPDFTeX
  \usepackage[T1]{fontenc}
  \usepackage[utf8]{inputenc}
  \usepackage{textcomp} % provide euro and other symbols
\else % if luatex or xetex
  \usepackage{unicode-math}
  \defaultfontfeatures{Scale=MatchLowercase}
  \defaultfontfeatures[\rmfamily]{Ligatures=TeX,Scale=1}
\fi
\usepackage{lmodern}
\ifPDFTeX\else  
    % xetex/luatex font selection
\fi
% Use upquote if available, for straight quotes in verbatim environments
\IfFileExists{upquote.sty}{\usepackage{upquote}}{}
\IfFileExists{microtype.sty}{% use microtype if available
  \usepackage[]{microtype}
  \UseMicrotypeSet[protrusion]{basicmath} % disable protrusion for tt fonts
}{}
\makeatletter
\@ifundefined{KOMAClassName}{% if non-KOMA class
  \IfFileExists{parskip.sty}{%
    \usepackage{parskip}
  }{% else
    \setlength{\parindent}{0pt}
    \setlength{\parskip}{6pt plus 2pt minus 1pt}}
}{% if KOMA class
  \KOMAoptions{parskip=half}}
\makeatother
\usepackage{xcolor}
\setlength{\emergencystretch}{3em} % prevent overfull lines
\setcounter{secnumdepth}{-\maxdimen} % remove section numbering
% Make \paragraph and \subparagraph free-standing
\ifx\paragraph\undefined\else
  \let\oldparagraph\paragraph
  \renewcommand{\paragraph}[1]{\oldparagraph{#1}\mbox{}}
\fi
\ifx\subparagraph\undefined\else
  \let\oldsubparagraph\subparagraph
  \renewcommand{\subparagraph}[1]{\oldsubparagraph{#1}\mbox{}}
\fi


\providecommand{\tightlist}{%
  \setlength{\itemsep}{0pt}\setlength{\parskip}{0pt}}\usepackage{longtable,booktabs,array}
\usepackage{calc} % for calculating minipage widths
% Correct order of tables after \paragraph or \subparagraph
\usepackage{etoolbox}
\makeatletter
\patchcmd\longtable{\par}{\if@noskipsec\mbox{}\fi\par}{}{}
\makeatother
% Allow footnotes in longtable head/foot
\IfFileExists{footnotehyper.sty}{\usepackage{footnotehyper}}{\usepackage{footnote}}
\makesavenoteenv{longtable}
\usepackage{graphicx}
\makeatletter
\def\maxwidth{\ifdim\Gin@nat@width>\linewidth\linewidth\else\Gin@nat@width\fi}
\def\maxheight{\ifdim\Gin@nat@height>\textheight\textheight\else\Gin@nat@height\fi}
\makeatother
% Scale images if necessary, so that they will not overflow the page
% margins by default, and it is still possible to overwrite the defaults
% using explicit options in \includegraphics[width, height, ...]{}
\setkeys{Gin}{width=\maxwidth,height=\maxheight,keepaspectratio}
% Set default figure placement to htbp
\makeatletter
\def\fps@figure{htbp}
\makeatother

% load packages
\usepackage{geometry}
\usepackage{xcolor}
\usepackage{eso-pic}
\usepackage{fancyhdr}
\usepackage{sectsty}
\usepackage{fontspec}
\usepackage{titlesec}

%% Set page size with a wider right margin
\geometry{a4paper, total={170mm,257mm}, left=20mm, top=20mm, bottom=20mm, right=50mm}

%% Let's define some colours
\definecolor{light}{HTML}{ECEFF4}
% \definecolor{light}{HTML}{ECEFF4}
\definecolor{highlight}{HTML}{81A1C1}
\definecolor{dark}{HTML}{2E3440}

\color{dark}

% Let's add the border on the right hand side 
\AddToShipoutPicture{% 
    \AtPageLowerLeft{% 
        \put(\LenToUnit{\dimexpr\paperwidth-3cm},0){% 
            \color{light!80}\rule{3cm}{\LenToUnit\paperheight}%
          }%
     }%
     % logo
    \AtPageLowerLeft{% start the bar at the bottom right of the page
        \put(\LenToUnit{\dimexpr\paperwidth-2cm},27.2cm){% move it to the top right
        \hfil{\Large\textbf{EC320}}\hfil% \color{light}\includegraphics[width=1.5cm]{_extensions/nrennie/PrettyPDF/uo-logo.png}
          }%
     }%
}

%% Style the page number
\fancypagestyle{mystyle}{
  \fancyhf{}
  \renewcommand\headrulewidth{0pt}
  \fancyfoot[R]{\texttt{\thepage}} % \texttt for monospaced font
  \fancyfootoffset{3.5cm}
}
\setlength{\footskip}{20pt}

%% style the chapter/section fonts
\chapterfont{\color{dark}\fontsize{20}{16.8}\selectfont}
\sectionfont{\color{dark}\fontsize{20}{16.8}\selectfont}
\subsectionfont{\color{dark}\fontsize{14}{16.8}\selectfont}
\titleformat{\subsection}
  {\sffamily\Large\bfseries}{\thesection}{1em}{}[{\titlerule[0.8pt]}]
  
% left align title
\makeatletter
\renewcommand{\maketitle}{\bgroup\setlength{\parindent}{0pt}
\begin{flushleft}
  {\sffamily\huge\textbf{\MakeUppercase{\@title}}} \vspace{0.3cm} \newline
  {\Large {\@subtitle}} \newline
  \@author
\end{flushleft}\egroup
}
\makeatother

%% Use some custom fonts
\setsansfont{FiraSansExtraCondensed}[
    Path=_extensions/nrennie/PrettyPDF/FiraSansExtraCondensed/,
    Scale=0.9,
    Extension = .ttf,
    UprightFont=*-ExtraLight,
    BoldFont=*-Regular,
    ItalicFont=*-Italic,
    ]

\setmainfont{FiraSansExtraCondensed}[
    Path=_extensions/nrennie/PrettyPDF/FiraSansExtraCondensed/,
    Scale=0.9,
    Extension = .ttf,
    UprightFont=*-ExtraLight,
    BoldFont=*-Regular,
    ItalicFont=*-Italic,
    ]

\setmonofont{JetBrainsMonoNerdFont}[
    Path=_extensions/nrennie/PrettyPDF/JetBrainsMonoNerdFont/,
    Scale=0.75,
    Extension = .ttf,
    UprightFont=*-Regular,
    BoldFont=*-Bold,
    ItalicFont=*-Italic,
]
\KOMAoption{captions}{tableheading}
\makeatletter
\@ifpackageloaded{caption}{}{\usepackage{caption}}
\AtBeginDocument{%
\ifdefined\contentsname
  \renewcommand*\contentsname{Table of contents}
\else
  \newcommand\contentsname{Table of contents}
\fi
\ifdefined\listfigurename
  \renewcommand*\listfigurename{List of Figures}
\else
  \newcommand\listfigurename{List of Figures}
\fi
\ifdefined\listtablename
  \renewcommand*\listtablename{List of Tables}
\else
  \newcommand\listtablename{List of Tables}
\fi
\ifdefined\figurename
  \renewcommand*\figurename{Figure}
\else
  \newcommand\figurename{Figure}
\fi
\ifdefined\tablename
  \renewcommand*\tablename{Table}
\else
  \newcommand\tablename{Table}
\fi
}
\@ifpackageloaded{float}{}{\usepackage{float}}
\floatstyle{ruled}
\@ifundefined{c@chapter}{\newfloat{codelisting}{h}{lop}}{\newfloat{codelisting}{h}{lop}[chapter]}
\floatname{codelisting}{Listing}
\newcommand*\listoflistings{\listof{codelisting}{List of Listings}}
\makeatother
\makeatletter
\makeatother
\makeatletter
\@ifpackageloaded{caption}{}{\usepackage{caption}}
\@ifpackageloaded{subcaption}{}{\usepackage{subcaption}}
\makeatother
\makeatletter
\@ifpackageloaded{tcolorbox}{}{\usepackage[skins,breakable]{tcolorbox}}
\makeatother
\makeatletter
\@ifundefined{shadecolor}{\definecolor{shadecolor}{rgb}{.97, .97, .97}}{}
\makeatother
\makeatletter
\@ifundefined{codebgcolor}{\definecolor{codebgcolor}{named}{light}}{}
\makeatother
\makeatletter
\ifdefined\Shaded\renewenvironment{Shaded}{\begin{tcolorbox}[frame hidden, boxrule=0pt, sharp corners, colback={codebgcolor}, breakable, enhanced]}{\end{tcolorbox}}\fi
\makeatother
\ifLuaTeX
  \usepackage{selnolig}  % disable illegal ligatures
\fi
\usepackage{bookmark}

\IfFileExists{xurl.sty}{\usepackage{xurl}}{} % add URL line breaks if available
\urlstyle{same} % disable monospaced font for URLs
\hypersetup{
  pdftitle={Introduction to Econometrics},
  pdfauthor={Dante Yasui},
  colorlinks=true,
  linkcolor={highlight},
  filecolor={Maroon},
  citecolor={Blue},
  urlcolor={highlight},
  pdfcreator={LaTeX via pandoc}}

\title{Introduction to Econometrics}
\usepackage{etoolbox}
\makeatletter
\providecommand{\subtitle}[1]{% add subtitle to \maketitle
  \apptocmd{\@title}{\par {\large #1 \par}}{}{}
}
\makeatother
\subtitle{Syllabus, Summer 2024}
\author{Dante Yasui}
\date{}

\begin{document}
\maketitle

\pagestyle{mystyle}

\section{Basics}\label{basics}

\begin{longtable}[]{@{}
  >{\raggedright\arraybackslash}p{(\columnwidth - 2\tabcolsep) * \real{0.2647}}
  >{\raggedright\arraybackslash}p{(\columnwidth - 2\tabcolsep) * \real{0.7353}}@{}}
\toprule\noalign{}
\begin{minipage}[b]{\linewidth}\raggedright
\end{minipage} & \begin{minipage}[b]{\linewidth}\raggedright
\textbf{Lecture}
\end{minipage} \\
\midrule\noalign{}
\endhead
\bottomrule\noalign{}
\endlastfoot
\textbf{lecture} & Recordings posted on Canvas \\
\textbf{office hours} &
\href{https://uoregon.zoom.us/j/98317465628?pwd=YNmQeo7AbiH39x9Vv11w1ACKsO0at9.1}{Tuesday:
10-11am} \\
&
\href{https://uoregon.zoom.us/j/95498739227?pwd=Txgm7ZalzCaZiipZMQZlQ1Uo2ewQ2a.1}{Thursday:
3-4pm} \\
\textbf{Lab} &
\href{https://uoregon.zoom.us/j/93947688014?pwd=szWAbYxRIbifx20m7iakzgYzcUN3zw.1}{Wednesday:
noon-2pm} \\
\textbf{materials} &
\href{http://smile.amazon.com/Introduction-Econometrics-Christopher-Dougherty/dp/0199676828/}{1.
Introduction to Econometrics} \\
&
\href{https://www.amazon.com/Mastering-Metrics-Path-Cause-Effect/dp/0691152845/}{2.
Mastering 'Metrics} \\
\textbf{repository} &
\href{https://github.com/dyasui/EC320Su24}{github.com/dyasui/EC320Su24} \\
\end{longtable}

\section{Course Summary}\label{course-summary}

This course introduces statistical techniques that economists use to
test economic theories and to estimate the relationships between
economic variables. Econometrics combines economics and statistics with
data to analyze and measure economic phenomena. In this class, we will
focus our attention on regression analysis--the workhorse of applied
econometrics. Using calculus and introductory statistics, we will
cultivate a working understanding of the theory underpinning regression
analysis, emphasizing the assumptions we must make to make causal
statements. Statistical programming is fundamental to practicing applied
econometrics. Thus we will teach the statistical programming language
\texttt{R} to apply insights from theory and learn how to work with
data. To the extent that you invest the requisite time and effort, you
can leave this course with marketable skills in data analysis and---most
importantly---a more sophisticated understanding of the notion that
\emph{correlation does not necessarily imply causation}.

\subsection{Software}\label{software}

\begin{itemize}
\tightlist
\item
  We will use the statistical programming language \texttt{R}.
\item
  We will use \textbf{RStudio} to interact with \texttt{R}.
\end{itemize}

Learning \texttt{R} is challenging, but well worth the effort.
\texttt{R} is a powerful and versatile tool for data analysis and
visualization, which makes it popular among employers. If you dedicate
the time and effort necessary to learn the language, you are likely to
reap a handsome return on the job market. I expect that you install
\texttt{R} and \textbf{RStudio} on your own computer. Don't worry, both
are free. I also recommend that you be thoughtful of how you choose to
organize your saved scripts, data, and assignments (e.g.~Home
\textgreater{} Documents \textgreater{} Classes \textgreater{} EC320).
For convenience, I make material available through Github.

\subsection{Textbooks and Other
Readings}\label{textbooks-and-other-readings}

\textbf{Econometrics books:} There are two required textbooks for this
course:

\begin{enumerate}
\def\labelenumi{\arabic{enumi}.}
\tightlist
\item
  \href{http://www.amazon.com/Introduction-Econometrics-Christopher-Dougherty/dp/0199676828/}{Introduction
  to Econometrics, 5\textsuperscript{th} ed.} by Christopher Dougherty
  (\textbf{ItE})
\item
  \href{https://www.amazon.com/Mastering-Metrics-Path-Cause-Effect/dp/0691152845/}{Mastering
  `Metrics: The Path from Cause to Effect} by Angrist and Pischke
  (\textbf{MM})
\end{enumerate}

You can purchase them at the UO duckstore or your preferred online
bookseller. I recommend that you read (or at least skim) the assigned
readings from the textbooks \emph{before} lecture. The lectures and the
readings are meant to \emph{complement} one another. The tentative
course schedule (below) lists the assigned readings for each topic.

\textbf{R books:} For learning \texttt{R}, a classic is Garrett
Grolemund and Hadley Wickham's \href{http://r4ds.had.co.nz}{R for Data
Science}. If you have previous experience coding in \texttt{R}, you may
want to check out Hadley Wickham's
\href{http://adv-r.had.co.nz/}{Advanced R}.

\subsection{Prerequisites:}\label{prerequisites}

Math 242 (Calculus) and Math 243 (Introduction to Statistics) or
equivalent.

\section{Labs, Assignments, and Exams}\label{labs-assignments-and-exams}

\subsection{Lab}\label{lab}

Although this class is officially asynchronous, I will also be offering
optional \emph{syncronous} lab sections. This will be a time scheduled
every week where you can join a Zoom meeting and be placed in breakout
rooms where you can work on assignments or quizes collaboratively with
other class members, or get direct help from me. I recommend
participating in these lab sessions if you are able. Between the amount
of math we used in this class and learning to code in \texttt{R}, there
will be a lot of potentially challenging material for you to tackle on
your own. Even if you are fairly comfortable with math and coding, I
have found that it can be rewarding to work with peers and explaining
things in your own words can help solidify your own understanding.

\begin{itemize}
\tightlist
\item
  While Lab sessions are optional, I will give an \textbf{extra credit
  point for every session you attend and actively particpate during.}
\end{itemize}

\subsection{Problem Sets and Koans}\label{problem-sets-and-koans}

Every week, you will have \emph{one problem set} as well as around
\emph{3\textasciitilde4 koans} and \emph{4 Canvas quizzes} to complete.

\textbf{Problem sets} will primarily focus on analytical problems but
may include a computational component. Submissions \textbf{must be your
own work}. You will receive \textbf{zero points} for copied work.

\begin{itemize}
\tightlist
\item
  Due on Friday midnight every week.
\item
  \textbf{PDF} and \textbf{html} are the only file types accepted for
  problem sets on Canvas
\item
  One file per problem set is permitted
\item
  Your lowest problem set score will be dropped
\end{itemize}

Feel free to work together on the assignments. Unless explicitly stated,
\textbf{each student is required to write and submit independent
answers}. This means that word-for-word copies will not be accepted and
will be viewed as academic dishonesty. In other words: You must place
answers \textbf{in your own words and code}. Copying from other people
(even if you worked with them) or from previous assignments is
considered cheating. Both will be submitted on Canvas under the
``Assignments'' tab.

\textbf{Koans} will be strictly focused on developing your computational
skills in R. They will be short and if you are actively keeping up with
learning R, it should not take long to complete them.

\begin{itemize}
\tightlist
\item
  All koans assigned for a given week will be due Friday midgight, but I
  recommend submitting one per day.
\item
  \textbf{html} is the only file type accepted for Koans on Canvas
\item
  Your lowest koan score will be dropped
\end{itemize}

Most of the Koans will be assigned in the first half of the course.
Toward the later half of the course, the problem sets will become more
dependent on R and Koans will be less frequent. These problem sets will
be longer and will have more time to complete them.

\textbf{Quizes} will be short multiple choice or numeric questions to
help you check your understanding.

\begin{itemize}
\tightlist
\item
  They will be graded based on completion only to make sure you are
  following along.
\end{itemize}

\textbf{Exams} will be timed, open-note, but individual and
closed-internet.

\begin{itemize}
\tightlist
\item
  You will have fixed amount of time to complete the quiz, but you can
  open it at any time from the Friday when it is posted to Sunday
  midnight.
\item
  You must submit a \emph{handwritten scan} of your work
\item
  Presentation matters. You may lose points for poor penmanship, scans,
  and presentation
\end{itemize}

\subsection{Late Policy}\label{late-policy}

\begin{itemize}
\tightlist
\item
  Late assignments will be accepted \textbf{up to 48 hours late} with a
  penalty of \textbf{2\% per hour late}
\item
  For example, when submitted 10 hours late, an assignment with a 90\%
  score would be penalized by 20\%, and the resulting final grade would
  be a 70\%
\item
  One koan and one problem set will be dropped at the end of the term
\end{itemize}

\subsection{Exams}\label{exams}

\begin{itemize}
\tightlist
\item
  The \textbf{Midterm} will be a timed Canvas quiz which will be
  released \textbf{Friday, July 5}. You will have 2 hours to complete,
  scan, and upload it.
\item
  The \textbf{Final} will be on \textbf{Friday, July 19, 2024}
\end{itemize}

Canvas exams will be open-note, but you will not be allowed to access
other internet resources. All work should be completed independently in
your own words.

\subsection{Makeup Assignments}\label{makeup-assignments}

I do not give makeup assignments. In extreme circumstances that lead you
to miss the midterm exam, I will consider re-weighting your grade toward
the final. To qualify for re-weighting, you will need to notify me no
later than two days after the exam.

\clearpage

\section{Grades}\label{grades}

Grades for this class will be assigned based on the following
assignments: weekly homework assignments, one midterm exam, one final
exam. Final grades will be determined based on your rank-ordered
position within the class (i.e., the course is curved)\footnote{The
  economics department has a uniform grading standard. In 300 and 400
  level classes, roughly 65\% of the class will receive A's and B's. I
  will not be able to tell you what your exact letter grade is at any
  point in time, because it depends on the scores of everyone else at
  the end of the course, but I will be able to give an assessment of
  your relative standing.}. The weights for the final grade:

\begin{longtable}[]{@{}ll@{}}
\toprule\noalign{}
Component & Percentage \\
\midrule\noalign{}
\endhead
\bottomrule\noalign{}
\endlastfoot
Koans & 10\% \\
Quizes & 10\% \\
Problem sets & 10\% \\
Midterm exam & 35\% \\
Final exam & 35\% \\
\end{longtable}

\emph{Note:} While assignments will be submitted on Canvas, due to any
potential curving of final grades, the gradebook on Canvas may not be
accurate---only an approximation. All adjustments of final grades will
be done in a local spreadsheet.

\section[Recommendations]{\texorpdfstring{Recommendations\footnote{Inspired
  from Professor Ed Rubin}}{Recommendations}}\label{recommendations}

\begin{enumerate}
\def\labelenumi{\arabic{enumi}.}
\item
  Be kind
\item
  \textbf{Take responsibility} for your own education and try to
  \textbf{learn} as much as you can
\item
  \textbf{Do your own work}
\item
  Develop your own \textbf{intuition}
\item
  Learn \texttt{R}. Struggle while you try and use Google or LLMs to
  figure things out
\item
  Come to \textbf{office hours}
\item
  Don't wait until the end of the term to ask for help
\item
  Start problem sets \textbf{early}---so you can come ask for help
\end{enumerate}

\subsection{Academic Integrity}\label{academic-integrity}

I will not tolerate cheating, plagiarism, and other violations of the
\href{https://studentlife.uoregon.edu/conduct}{Student Conduct Code}. If
you are caught cheating or plagiarizing on any component of this course,
you will receive a failing grade for the term and I will report your
offense to the university.

\subsection{Accommodations}\label{accommodations}

Notify me if there are aspects of this course that pose
disability-related barriers to your participation. If you require
special accommodations for a documented disability, then you will need
to provide me a letter from the
\href{https://aec.uoregon.edu/}{Accessible Education Center} (AEC) that
verifies your need and details the appropriate accommodations. Please
make arrangements with the AEC by the end of Week 1. If your
accommodations include exam proctoring at the AEC, then you are
responsible for scheduling those exams with the AEC \emph{at least seven
days in advance}.

\subsection{Acknowledgements}\label{acknowledgements}

This class was originally created by
\href{https://ajdickinson.github.io/}{Andrew Dickinson}. All class
materials for this class were forked from his original repository
\href{https://github.com/ajdickinson/EC320S24}{here}. His original
acknowledgment section is replicated below:

Material for this course has contributions from
\href{http://edrub.in/}{Ed Rubin}
(\href{https://github.com/edrubin}{@edrubin}),
\href{https://kyleraze.com/}{Kyle Raze}
(\href{https://github.com/kyleraze}{@kyleraze}),
\href{https://philip-economides.com/}{Philip Economides}
(\href{https://github.com/peconomi}{@peconomi}), and
\href{https://www.emmettsaulnier.com/}{Emmett Saulnier}, who have taught
the class prior to me and graciously made their work public and inspired
me to pay it forward. Additionally I source some material from
\href{https://nickchk.com/}{Nick Huntington-Klein}
(\href{https://github.com/NickCH-K}{@NickCH-K}), who maintains a
\href{https://nickchk.com/causalgraphs.html}{trove of resources} for
learning causal inference, and
\href{https://ben-lambert.com/econometrics/}{Ben Lambert} and his
\href{.https://www.youtube.com/playlist?list=PLwJRxp3blEvZyQBTTOMFRP_TDaSdly3gU}{undergradaute
course in econometrics} that helped me learn this material as a student
and to teach this material as an instructor. Finally, I would like to
thank \href{https://cobriant.github.io/}{Colleen O'Briant}
(\href{https://github.com/cobriant}{@cobriant}) for making available her
\href{https://github.com/cobriant/tidyverse_koans}{tidyverse\_koans}
materials for use in this course.

HTML slides are generated using the \href{https://quarto.org/}{quarto}.
Source code for the slides is available in the ``slides'' directory of
this repository. PDFs of the slides are generated using the
\texttt{renderthis} package in R. PDF Documents are compiled in quarto
and using the wonderful quarto extension
\href{https://github.com/nrennie/PrettyPDF}{PrettyPDF} by
\href{https://nrennie.rbind.io/}{Nicola Rennie}.

\clearpage

\section{Tentative Schedule}\label{tentative-schedule}

\begin{longtable}[]{@{}
  >{\raggedright\arraybackslash}p{(\columnwidth - 6\tabcolsep) * \real{0.1215}}
  >{\raggedright\arraybackslash}p{(\columnwidth - 6\tabcolsep) * \real{0.3925}}
  >{\raggedright\arraybackslash}p{(\columnwidth - 6\tabcolsep) * \real{0.2710}}
  >{\raggedright\arraybackslash}p{(\columnwidth - 6\tabcolsep) * \real{0.2150}}@{}}
\toprule\noalign{}
\begin{minipage}[b]{\linewidth}\raggedright
\textbf{Week}
\end{minipage} & \begin{minipage}[b]{\linewidth}\raggedright
\textbf{Lectures}
\end{minipage} & \begin{minipage}[b]{\linewidth}\raggedright
\textbf{Readings}
\end{minipage} & \begin{minipage}[b]{\linewidth}\raggedright
\textbf{Assignments}
\end{minipage} \\
\midrule\noalign{}
\endhead
\bottomrule\noalign{}
\endlastfoot
\texttt{1} & Introduction & Syllabus & Install R and RStudio \\
\texttt{1} & Stats Review & ItE Review, MM 1 (appendix) & K01, 02, 03 \\
\texttt{1} & Fundamental Econometric Problem & MM 1 & HW 1 \\
\texttt{1} & The Logic of Regression & MM 2 & Quiz 1 \\
\texttt{2} & Linear Regression: Estimation & ItE 1 & K04, 05, 06, 06b \\
\texttt{2} & Classical Assumptions & ItE 2 & HW 2 \\
\texttt{2} & Linear Regression: Inference & ItE 2 & Quiz 2 \\
\texttt{2} & Linear Regression: Estimation & ItE 2 & \emph{Midterm} \\
\texttt{3} & Multiple Linear Regression: Estimation & ItE 3, 6.2; MM 2
(appendix) & K07, 08, 09, 10 \\
\texttt{3} & Multiple Linear Regression: Inference & ItE 3 & HW 3 \\
\texttt{3} & Nonlinear Relationships & ItE 4 & Quiz 3 \\
\texttt{4} & Qualitative Variables & ItE 5 & K11, 12, 13 \\
\texttt{4} & Model Specification & ItE 6 & HW 4 \\
\texttt{4} & Heteroskedasticity & ItE 7 & Quiz 4 \\
\texttt{4} & Final Review & Review Sheet & \emph{Final Exam} \\
\end{longtable}



\end{document}
